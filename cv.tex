%%%%%%%%%%%%%%%%%%%%%%%%%%%%%%%%%%%%%%%%%%%%%%%%%%%%%%%%%%%%%%%%%%%%%%%%
%%%%%%%%%%%%%%%%%%%%%% Simple LaTeX CV Template %%%%%%%%%%%%%%%%%%%%%%%%
%%%%%%%%%%%%%%%%%%%%%%%%%%%%%%%%%%%%%%%%%%%%%%%%%%%%%%%%%%%%%%%%%%%%%%%%

%%%%%%%%%%%%%%%%%%%%%%%%%%%%%%%%%%%%%%%%%%%%%%%%%%%%%%%%%%%%%%%%%%%%%%%%
%% NOTE: If you find that it says                                     %%
%%                                                                    %%
%%                           1 of ??                                  %%
%%                                                                    %%
%% at the bottom of your first page, this means that the AUX file     %%
%% was not available when you ran LaTeX on this source. Simply RERUN  %%
%% LaTeX to get the ``??'' replaced with the number of the last page  %%
%% of the document. The AUX file will be generated on the first run   %%
%% of LaTeX and used on the second run to fill in all of the          %%
%% references.                                                        %%
%%%%%%%%%%%%%%%%%%%%%%%%%%%%%%%%%%%%%%%%%%%%%%%%%%%%%%%%%%%%%%%%%%%%%%%%

%%%%%%%%%%%%%%%%%%%%%%%%%%%% Document Setup %%%%%%%%%%%%%%%%%%%%%%%%%%%%

% Don't like 10pt? Try 11pt or 12pt
\documentclass[10pt]{article}

% This is a helpful package that puts math inside length specifications
\usepackage{calc, graphicx}
\usepackage{polski}
\usepackage[utf8]{inputenc}
\usepackage{multirow}

\usepackage{hyperref}

\hypersetup{
    bookmarks=true,         % show bookmarks bar?
    unicode=false,          % non-Latin characters in Acrobat’s bookmarks
    pdftoolbar=true,        % show Acrobat’s toolbar?
    pdfmenubar=true,        % show Acrobat’s menu?
    pdffitwindow=false,     % window fit to page when opened
    pdfstartview={FitH},    % fits the width of the page to the window
    pdftitle={My title},    % title
    pdfauthor={Author},     % author
    pdfsubject={Subject},   % subject of the document
    pdfcreator={Creator},   % creator of the document
    pdfproducer={Producer}, % producer of the document
    pdfkeywords={keywords}, % list of keywords
    pdfnewwindow=true,      % links in new window
    colorlinks=true,       % false: boxed links; true: colored links
    linkcolor=red,          % color of internal links
    citecolor=green,        % color of links to bibliography
    filecolor=magenta,      % color of file links
    urlcolor=black          % color of external links
}


% Simpler bibsection for CV sections
% (thanks to natbib for inspiration)
\makeatletter
\newlength{\bibhang}
\setlength{\bibhang}{1em}
\newlength{\bibsep}
 {\@listi \global\bibsep\itemsep \global\advance\bibsep by\parsep}
\newenvironment{bibsection}
    {\minipage[t]{\linewidth}\list{}{%
        \setlength{\leftmargin}{\bibhang}%
        \setlength{\itemindent}{-\leftmargin}%
        \setlength{\itemsep}{\bibsep}%
        \setlength{\parsep}{\z@}%
        }}
    {\endlist\endminipage}
\makeatother

% Layout: Puts the section titles on left side of page
\reversemarginpar

%
%         PAPER SIZE, PAGE NUMBER, AND DOCUMENT LAYOUT NOTES:
%
% The next \usepackage line changes the layout for CV style section
% headings as marginal notes. It also sets up the paper size as either
% letter or A4. By default, letter was used. If A4 paper is desired,
% comment out the letterpaper lines and uncomment the a4paper lines.
%
% As you can see, the margin widths and section title widths can be
% easily adjusted.
%
% ALSO: Notice that the includefoot option can be commented OUT in order
% to put the PAGE NUMBER *IN* the bottom margin. This will make the
% effective text area larger.
%
% IF YOU WISH TO REMOVE THE ``of LASTPAGE'' next to each page number,
% see the note about the +LP and -LP lines below. Comment out the +LP
% and uncomment the -LP.
%
% IF YOU WISH TO REMOVE PAGE NUMBERS, be sure that the includefoot line
% is uncommented and ALSO uncomment the \pagestyle{empty} a few lines
% below.
%

%% Use these lines for letter-sized paper
\usepackage[paper=letterpaper,
            %includefoot, % Uncomment to put page number above margin
            marginparwidth=1.2in,     % Length of section titles
            marginparsep=.05in,       % Space between titles and text
            margin=1in,               % 1 inch margins
            includemp]{geometry}

%% Use these lines for A4-sized paper
%\usepackage[paper=a4paper,
%            %includefoot, % Uncomment to put page number above margin
%            marginparwidth=30.5mm,    % Length of section titles
%            marginparsep=1.5mm,       % Space between titles and text
%            margin=25mm,              % 25mm margins
%            includemp]{geometry}

%% More layout: Get rid of indenting throughout entire document
\setlength{\parindent}{0in}

%% This gives us fun enumeration environments. compactitem will be nice.
\usepackage{paralist}

%% Reference the last page in the page number
%
% NOTE: comment the +LP line and uncomment the -LP line to have page
%       numbers without the ``of ##'' last page reference)
%
% NOTE: uncomment the \pagestyle{empty} line to get rid of all page
%       numbers (make sure includefoot is commented out above)
%
\usepackage{fancyhdr,lastpage}
\pagestyle{fancy}
%\pagestyle{empty}      % Uncomment this to get rid of page numbers
\fancyhf{}\renewcommand{\headrulewidth}{0pt}
\fancyfootoffset{\marginparsep+\marginparwidth}
\newlength{\footpageshift}
\setlength{\footpageshift}
          {0.5\textwidth+0.5\marginparsep+0.5\marginparwidth-2in}
\lfoot{\hspace{\footpageshift}%
       \parbox{4in}{\, \hfill %
                    %\arabic{page} of \protect\pageref*{LastPage} % +LP
                    \arabic{page}                               % -LP
                    \hfill \,}}

% Finally, give us PDF bookmarks
\usepackage{color,hyperref}
\definecolor{darkblue}{rgb}{0.0,0.0,0.3}
\hypersetup{colorlinks,breaklinks,
            linkcolor=darkblue,urlcolor=darkblue,
            anchorcolor=darkblue,citecolor=darkblue}

%%%%%%%%%%%%%%%%%%%%%%%% End Document Setup %%%%%%%%%%%%%%%%%%%%%%%%%%%%


%%%%%%%%%%%%%%%%%%%%%%%%%%% Helper Commands %%%%%%%%%%%%%%%%%%%%%%%%%%%%

% The title (name) with a horizontal rule under it
%
% Usage: \makeheading{name}
%
% Place at top of document. It should be the first thing.
\newcommand{\makeheading}[1]%
        {\hspace*{-\marginparsep minus \marginparwidth}%
         \begin{minipage}[t]{\textwidth+\marginparwidth+\marginparsep}%
                {\large \bfseries #1}\\[-0.15\baselineskip]%
                 \rule{\columnwidth}{1pt}%
         \end{minipage}}

% The section headings
%
% Usage: \section{section name}
%
% Follow this section IMMEDIATELY with the first line of the section
% text. Do not put whitespace in between. That is, do this:
%
%       \section{My Information}
%       Here is my information.
%
% and NOT this:
%
%       \section{My Information}
%
%       Here is my information.
%
% Otherwise the top of the section header will not line up with the top
% of the section. Of course, using a single comment character (%) on
% empty lines allows for the function of the first example with the
% readability of the second example.
\renewcommand{\section}[2]%
        {\pagebreak[2]\vspace{1.3\baselineskip}%
         \phantomsection\addcontentsline{toc}{section}{#1}%
         \hspace{0in}%
         \marginpar{
         \raggedright \scshape #1}#2}

% An itemize-style list with lots of space between items
\newenvironment{outerlist}[1][\enskip\textbullet]%
        {\begin{itemize}[#1]}{\end{itemize}%
         \vspace{-.6\baselineskip}}

% An environment IDENTICAL to outerlist that has better pre-list spacing
% when used as the first thing in a \section
\newenvironment{lonelist}[1][\enskip\textbullet]%
        {\vspace{-\baselineskip}\begin{list}{#1}{%
        \setlength{\partopsep}{0pt}%
        \setlength{\topsep}{0pt}}}
        {\end{list}\vspace{-.6\baselineskip}}

% An itemize-style list with little space between items
\newenvironment{innerlist}[1][\enskip\textbullet]%
        {\begin{compactitem}[#1]}{\end{compactitem}}

% To add some paragraph space between lines.
% This also tells LaTeX to preferably break a page on one of these gaps
% if there is a needed pagebreak nearby.
\newcommand{\blankline}{\quad\pagebreak[2]}

% 

%%%%%%%%%%%%%%%%%%%%%%%% End Helper Commands %%%%%%%%%%%%%%%%%%%%%%%%%%%

%%%%%%%%%%%%%%%%%%%%%%%%% Begin CV Document %%%%%%%%%%%%%%%%%%%%%%%%%%%%

\begin{document}
\makeheading{Piotr Szachewicz}

\section{}
%
% NOTE: Mind where the & separators and \\ breaks are in the following
%       table.
%
% ALSO: \rcollength is the width of the right column of the table
%       (adjust it to your liking; default is 1.85in).
%
\newlength{\rcollength}\setlength{\rcollength}{1.45in}%
%

\begin{tabular}[t]{@{}p{\textwidth-\rcollength}p{\rcollength}}
  e-mail: \href{mailto:piotr.szachewicz@gmail.com}{piotr.szachewicz@gmail.com} & \multirow{3}{*}{\includegraphics[scale=0.18]{photo.jpg}}\\
  github: \href{https://github.com/piotr-szachewicz}{https://github.com/piotr-szachewicz}\\
\end{tabular}

%\section{Date of birth}
%18 August 1986
%\\

\section{Education}

\textbf{Poznań University of Technology} (Poland)
	\begin{outerlist}
		\item[] MSc Eng in Computer Science (February 2012 - July 2013)
		\item[] Specialization: Intelligent Decision Support Systems\\
	\end{outerlist}

\textbf{Universitat Roviri i Virgili} (Spain)
	\begin{outerlist}
		\item[] Erasmus exchange (September 2012 - February 2013)\\
	\end{outerlist}
	
\textbf{Poznań University of Technology} (Poland)
	\begin{outerlist}
		\item[] BSc Eng in Computer Science (October 2008 - February 2012)\\
	\end{outerlist}

\textbf{Adam Mickiewicz University, Faculty of Physics} (Poland)

	\begin{outerlist}
		\item[] BSc in Acoustics/Sound Engineering (October 2005 - June 2008)
	\end{outerlist}

\section{Experience}

\textbf{March 2020 - present} -- Spotify -- \textbf{Senior Engineer} (London).
	\begin{outerlist}
		\item[] \textbf{Product:} Content Compliance systems, proactive detection of policy violating content, systems to review flagged content.
		\item[] \textbf{Technologies:} Java, GKE, Salesforce
		\item[] \textbf{Other responsibilities:} driving design discussions with stakeholders and other teams, writing RFCs, interviewing, helping to onboard new colleagues, sponsoring my colleagues, helping to organize internal conferences, mentor in an internal Spotify mentoring program, member of a tribe Technical Steering Group.\\
	\end{outerlist}



\textbf{December 2018 - February 2020} -- Spotify -- \textbf{Software Engineer} (London).
	\begin{outerlist}
		\item[] Relocated to London to continue working on the Product Catalog with a newly hired team.\\
	\end{outerlist}

\textbf{July 2017 - November 2018} -- Spotify -- \textbf{Software Engineer} (Stockholm).

	\begin{outerlist}
		\item[] \textbf{Product:} Product Catalog - internal system to configure products and offers launched by Spotify in different markets.
		\item[] \textbf{Technologies:} Python, Java\\
	\end{outerlist}

\textbf{October 2014 - April 2017} -- Sage -- \textbf{Senior Ruby on Rails Developer}\\(Barcelona, Spain).

\begin{itemize}
  \item \textbf{Product:} \href{http://uk.sageone.com/}{Sage Business Cloud Accounting} -- online accounting software.
  \item \textbf{Tools:} Ruby on Rails, React.\\
\end{itemize}

\textbf{January 2013 - September 2014} -- Titanis -- \textbf{Python Developer} (Poznań, Poland).

\begin{itemize}
 \item \textbf{Product:} \href{http://neuro-forma.com}{Neuroforma}, neurorehabilitation software with elements of virtual reality.
 \item \textbf{Tools:} Python, Django, CoffeeScript.
\end{itemize}

%\pagebreak

%\textbf{July 2011 - September 2011} -- Titanis -- \textbf{C++ Developer}.
%\begin{itemize}
%  \item Product: Eyetracker, software for a head-mounted eyetracker.
%  \item Tools: C++, openFrameworks, video image processing.
  %\item \url{http://git.nimitz.pl/eyetracker.git}
%\end{itemize}

\textbf{May 2011 - September 2011} -- Alliance Technology -- \textbf{Java EE Developer}.
\begin{itemize}
  \item \textbf{Product:} banking software.
  \item \textbf{Tools:} Java EE, Hibernate.
\end{itemize}

\textbf{August 2010 - January 2013} -- Titanis -- \textbf{Java Developer}.
\begin{itemize}
  \item \textbf{Product:} \href{https://github.com/BrainTech/svarog}{Svarog} -- EEG signal viewer, annotator, analyzer and recorder.
  \item Tools: Java, Maven, digital signal processing.
\end{itemize}

\textbf{October 2009 - January 2010} -- Adam Mickiewicz University -- \textbf{Teaching Assistant}.
\begin{itemize}
 \item Human-Computer Interaction laboratory practical classes.
\end{itemize}

%\item[] \textbf{2006 - 2009} -- sound engineer in Blue Note Jazz Club, Poznań.

\section{Other projects}

\textbf{March 2012 - November 2013} -- Poznan University of Technology -- publication.
\begin{itemize} 
  \item \textbf{Project:} \href{http://goo.gl/Q9VUsw}{Automatic species counterpoint composition by means of the dominance relation}.
  \item \textbf{Tools:} Java, Python, Latex.
  \item Published in \href{http://goo.gl/Q9VUsw}{"Journal of Mathematics and Music"}, 2015.
\end{itemize}

\textbf{June 2012 - June 2013} -- Poznan University of Technology -- \textbf{MSc Eng thesis}.
\begin{itemize}
  \item \textbf{Project:} \href{https://github.com/piotr-szachewicz/event-related-desynchronization}{Classification of Motor Imagery for Brain-Computer Interfaces.}
  \item \textbf{Tools:} Python, Machine Learning, EEG, DSP.
\end{itemize}

\textbf{September 2011 - January 2012} -- Poznan University of Technology -- \textbf{BSc Eng~thesis}.
\begin{itemize}
  \item \textbf{Project:} developing scholarship module for Sokrates 2 -- a computer system for managing teaching at universities created by the Poznań University of Technology.
  \item \textbf{Tools:} Java EE, Maven, Apache CXF, Adobe Flex, Hibernate, Oracle Database.
\end{itemize}



%\section{Conferences/Papers}

%\begin{itemize}

%\item[] \textbf{2013} -- M. Komosiński, P. Szachewicz, ``Automatic species counterpoint composition using the dominance relation'' (paper; in progress)

%\item[] \textbf{June 2012} -- Concepts-thinking-information, Poznań -- ``Automatic counterpoint composition' (speech).

%\item[] \textbf{December 2009} -- 5th Poznań Cognitive Science Forum (5. Poznańskie Forum Kognitywistyczne) -- ``Computational aesthetics. Methods for automatical aesthetic value evaluation.'' (speech + paper).

%\item[] \textbf{January 2009} -- 4th Poznań Cognitive Science Forum -- ``How to build an artificial artist? Review of approaches to artificial art generation.'' (speech + paper).

%\end{itemize}

%\section{Skills}

%web-related:
%\begin{itemize}
% \item CoffeeScript, jQuery, jQuery UI, Django, RESTful WebServices, PHP
%\end{itemize}

%Java related -- Java, Java Enterprise Edition, JUnit, Maven \\
%databases --- Relational Databases, ORM (Hibernate, Django ORM)
%version control --- Git, Mercurial, Subversion
%other --- Linux, Python, Digital Signal Processing, Electroencephalography, LaTeX, Artificial Intelligence, Machine Learning, Octave, Matlab, OOP, Scrum

% C++, NoSQL, Hadoop, Erlang, C\#

\section{Languages} Polish, English, Spanish

\section{Personal interests} \href{http://sh-ch.band/}{SH-CH} -- instrumental music project

\end{document}

%%%%%%%%%%%%%%%%%%%%%%%%%% End CV Document %%%%%%%%%%%%%%%%%%%%%%%%%%%%%
